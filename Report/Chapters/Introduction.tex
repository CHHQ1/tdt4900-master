\chapter{Introduction}

In recent years, digital currencies know as ``cryptocurrencies'' have become popular
and most popular of these is Bitcoin. The term ``cryptocurrency'' stems from the fact
that a cryptographic hash function is at the core of the algorithms involved. For bitcoin,
the SHA-256 hash algorithm is used in a two-pass configuration producing a double SHA-256
hash \cite{bitcoin}, although other currencies uses other algorithms.

Using bitcoin mining as an application, this project develops a SHA-256 accelerator
for the Single-ISA Heterogeneous MAny-core Computer, SHMAC. To achieve higher throughputs
and energy efficiency, a DMA is also developed. The two modules are integrated into a
general-purpose CPU tile, and an estimation of the bitcoin mining efficiency is made
using a benchmark application measuring the performance of the double SHA-256 calculation
that is central to the bitcoin mining algorithm. Energy efficiency is also estimated by
measuring the power used when running the benchmark.

\section{Original Assignment Text}

This project aims to develop a bitcoin mining accelerator that will ultimately be
used in the single-ISA, many-core, heterogeneous computing platform SHMAC. Bitcoin
mining is, at its core, a SHA256 hashing problem, so part of the assignment will
be to keep the interface generic enough such that other cryptographic algorithms
can be readily developed.

This part of the project will focus on building an end-to-end and fully functional
hardware and software system that is able to fully participate as a miner in the
network (that is, mine Bitcoins and communicate with peers).

The energy efficiency of the hardware implementation should be evaluated against that
of a general-purpose CPU and should constitute a major aspect of the report.

\section{Comments on the Assignment Text}

Recent hardware developments within the field of bitcoin mining, most important of
which is the introduction of dedicated ASIC mining chips described in section \ref{sec:asic},
have led to bitcoin mining using CPUs, GPUs and many FPGA-based designs being considered
unprofitable and wasteful in terms of the energy expended.

In such an environment, it is doubtful that a FPGA-based SHMAC implementation can compete
with ASIC designs. The assignment then basically boils down to: is bitcoin mining possible using SHMAC?
Does the mining process benefit from heterogenity, and what performance and energy
efficiency can be expected when using SHMAC as a bitcoin miner?

Another relevant and interesting issue is how the system scales; given enough processing
cores, could SHMAC compare to dedicated bitcoin mining systems in power efficiency or
computing power?

%\section{Structure of the report}
%\todo{It is common in many project reports to introduce the rest of the chapters in the final part of the Introduction chapter. Fill this one out before deadline.}

