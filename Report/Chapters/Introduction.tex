\chapter{Introduction}

In recent years, digital currencies know as ``cryptocurrencies'' have become popular
and most popular of these is bitcoin. The term ``cryptocurrency'' stems from the fact
that a cryptographic hash function is at the core of the algorithms involved. For bitcoin,
the SHA-256 hash algorithm is used in a two-pass configuration producing a double SHA-256
hash \cite{bitcoin}.

Producing new bitcoins is done through a process called mining, described in Section \ref{sec:bitcoin-mining}.
Mining requires large amounts of computing power, and has led to a quick development in
the hardware used in the process, from regular CPUs and graphics processors, to Field-Programmable Gate Arrays (FPGA) and highly specialized Application Specific
Integrated Circuits (ASIC). This evolution of mining hardware is described in Section \ref{sec:bitcoin-history}.
Bitcoin mining requires a large degree of thread-level parallelism, because the SHA-256
algorithm does not provide any opportunities to exploit parallelism when calculating a single hash;
instead, running several separate SHA-256 computations in parallel provides a possibility for
exploiting parallelism to increase the number of SHA-256 computations that can be run in a specified period of time.

Using bitcoin mining as an application, this project develops a SHA-256 accelerator
for the Single-ISA Heterogeneous MAny-core Computer, SHMAC, described in section \ref{sec:shmac},
which provides a high degree of thread-level parallelism. To achieve higher throughput for the
SHA-256 accelerator and the possibility of higher energy efficiency overall, a DMA is also
developed. The two modules are integrated into a general-purpose CPU tile, and an estimation
of the bitcoin mining efficiency is made using a benchmark application measuring the performance
of the system by measuring how many hashes can be calculated each second and how much power
is used when doing the calculations.

\section{Original Assignment Text}

This project aims to develop a bitcoin mining accelerator that will ultimately be
used in the single-ISA, many-core, heterogeneous computing platform SHMAC. Bitcoin
mining is, at its core, a SHA-256 hashing problem, so part of the assignment will
be to keep the interface generic enough such that other cryptographic algorithms
can be readily developed.

This part of the project will focus on building an end-to-end and fully functional
hardware and software system that is able to fully participate as a miner in the
network (that is, mine bitcoins and communicate with peers).

The energy efficiency of the hardware implementation should be evaluated against that
of a general-purpose CPU and should constitute a major aspect of the report.

\section{Comments on the Assignment Text}

Recent hardware developments within the field of bitcoin mining, most important of
which is the introduction of dedicated ASIC mining chips described in Section \ref{sec:asic},
have led to bitcoin mining using CPUs, GPUs and many FPGA-based designs being considered
unprofitable and wasteful in terms of the energy expended.

In such an environment, it is doubtful that an FPGA-based SHMAC implementation can compete
with ASIC designs. The assignment then basically boils down to: is bitcoin mining possible using SHMAC?
Does the mining process benefit from heterogenity, and what performance and energy
efficiency can be expected when using SHMAC as a bitcoin miner? How does it compare to
existing FPGA-based bitcoin miners.

%\section{Structure of the report}
%\todo{It is common in many project reports to introduce the rest of the chapters in the final part of the Introduction chapter. Fill this one out before deadline.}

