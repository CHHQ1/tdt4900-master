\chapter{Conclusion and Future Work}
\label{cha:conclusion}

The results obtained with the benchmark shows that using the SHA-256 accelerator and DMA
module described in this report, it is possible to get a bitcoin mining performance of 175,7~kH/s when running the
hashing benchmark on 14 cores simultaneously while using the DMA and a performance of 179,5~kH/s when
not using the DMA. The best energy efficiency is obtained when running the benchmark on 14 cores
\emph{without} the DMA enabled, at 163,2~kH/J.

Although the performance and energy efficiency obtained in this study cannot compete
with other bitcoin mining solutions implemented on FPGAs, providing 128 times lower energy efficiency than
a dedicated bitcoin mining system, it still shows that SHMAC is an excellent platform
for exploiting thread-level parallelism, and using our hardware accelerators provided a
near-linear performance scaling over 14 CPU cores while retaining good energy efficiency.

With respect to bitcoin mining, the days of FPGAs are gone, and bitcoin mining is now
the domain of highly optimized ASICs. However, although SHMAC cannot be used for bitcoin
mining, the high degree of thread-level parallelism provided by the platform can be well exploited
in other application areas benefitting from high levels of parallelism.

\section{Enhancing the SHA-256 Module}

The SHA-256 module can obtain performance increases from pipelining. Although this uses more
resources on the FPGA chip as well as requiring additional memory transfers from a CPU or a
DMA to keep it fed with data, it can increase performance and energy efficiency depending
on the application.

\section{Enhancing the DMA Module}

The current DMA Module only supports transfers of single 32-bit words, while the interconnect used in SHMAC supports 128-bit words.
This means four transfers are done on the network when only one is needed. Exploiting the unused bandwidth could
provide a speedup of as much as 75~\% and reduce the workload on the DMA significantly.
While expanding the DMA transfer width, it must still be made compatible with the other modules on the tile, including the SHA-256 accelerator.
This likely requires a solution where individual 32-bit words may be selected and written individually.

The DMA could also be expanded to include support for wishbone burst transfers. This can allow quicker data
transfer, as overhead is reduced, and potentially improving throughput.


%The reason to not expand to 128-bit blocks per individual transfer was due to the concern of forced alignment within the blocks.
%While transferring aligned blocks of data from one memory location to another is \todo{Yaman said this. Source needed}common, switching a 32-bit word's position inside a block would not be possible for the DMA, if we were to expand the size without making considerable change to the DMA. 
%We were concerned that this could prevent us from transferring data between neighbour registers on the tile.
%Furthermore, it was concidered outside the scope of the project to further enhance the DMA Module, as the main idea was that this would enhance SHMAC generally, but not the hashing module.
%In order to focus remaining project time on the Hashing module, and to have the option of writing to any register inside the tile, 32-bit data transfer were therefore chosen.
%
% %**NOTE: The above (and the stuff below) should be in architecture, as it explains an architectural choice!
%
%Expanding the data registers and lines to 128 bits is straightforward, but if it is desirable to store 32-bit words to new block positions, then the DMA itself must be changed to account for this.
%DMA Module must be changed with smarter logic, that can put together new 128-bit blocks for storing, with the fetched data.
%Alternatively, the system may contain two DMA Modules: One for 32-bit transfers that can access the entire address space, and one for 128-bit transfers that will be most efficient for data transfers when alignment is not a problem.

%The DMA Module only supports classic single cycle transfers.
%If burst mode is to be supported, both the DMA WB Master and the channel arbiter must be modified to allow burst mode.
%In the current system, the arbiter it will select data alternatively from channels 0 and 1 if both are active, which may prevent burst mode, and the WB Master does only 1 request per transfer cycle.     

%There is also another option: The WISHBONE Public Domain Library for VHDL offers a DMA Module of its own.
%It is intended only for educational and benchmarking purposes, and is thus very simple, but it can be adapted for more serious work.
%It supports both single transfers and burst transfers.
%However, it only does 32-bit transfers, thus it has the same issues as the DMA Module we have implemented, regarding the 128-bit bus.
%Its initiation is vaguely described in the Library manual, and there are no description of any slave interface, which implies that adapting this module will require expantion with a slave interface, similar to the one implemented for this project, in order to function on SHMAC
%
% %**NOTE: Why suggest a concrete alternative? Rather suggest just an open source DMA or (better) don't suggest anything at all.
%\cite{WBLibrary}.

