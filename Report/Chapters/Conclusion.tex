\chapter{Conclusion}
\label{cha:conclusion}

%NOTE TO SELF: Double check if there is sufficient material provided with this conclusion
The results presented in this report shows that although SHMAC is not powerful
enough to participate in the bitcoin network, an accelerator provides a huge
performance boost to the calculation of SHA256 hashes and can be used in other
applications.

Bitcoin mining is currently an activity that requires a tremendous amount of computing power.
As such, it is no surprise that ASICs rule this domain.

\section{Future work}
\todo{Option: Create a chapter named "Discussion", and move future work there, before conclusion?}

\subsection{Enhancing the DMA Module}

The current DMA Module only supports transfer of 32-bit words, which means four transfers for each 128-bit blocks used by SHMAC.
This hampers potential performance severely, as up to 75\% of the DMA's work could have been avoided.
When using the current DMA Module with the SHA-256 hashing, the performance gain is at only \textbf{XXX} percent, while the increase in energy usage is in average about \todo{Due to lack of data, we cannot make the outrageous cry of how bad it is, until we know how bad it is} \textbf{XXX}.

\textbf{Insert data about how bad or slightly good this is, when calculations are done.} 

The reason to not expand to 128-bit blocks per individual transfer was due to the concern of forced alignment within the blocks.
While transferring aligned blocks of data from one memory location to another is \todo{Yaman said this. Source needed}common, switching a 32-bit word's position inside a block would not be possible for the DMA, if we were to expand the size without making considerable change to the DMA. 
We were concerned that this could prevent us from transferring data between neighbour registers on the tile.
Furthermore, it was concidered outside the scope of the project to further enhance the DMA Module, as the main idea was that this would enhance SHMAC generally, but not the hashing module.
In order to focus remaining project time on the Hashing module, and to have the option of writing to any register inside the tile, 32-bit data transfer were therefore chosen.

Nevertheless, we recommend enhancing the DMA Module for 128-bit transfers, since SHMAC as a whole will benefit greatly on this, reducing the work down to a fourth.
Not only may total throughput increase, but the DMA Module has less overhead in transferring data, compared to a CPU that does branching to control its own data transfer.

Expanding the data registers and lines to 128 bits is straightforward, but if it is desirable to store 32-bit words to new block positions, then the DMA itself must be changed to account for this.
DMA Module must be changed with smarter logic, that can put together new 128-bit blocks for storing, with the fetched data.
Alternatively, the system may contain two DMA Modules: One for 32-bit transfers that can access the entire address space, and one for 128-bit transfers that will be most efficient for data transfers when alignment is not a problem.

The DMA Module only supports classic single cycle transfers.
If burst mode is to be supported, both the DMA WB Master and the channel arbiter must be modified to allow burst mode.
In the current system, the arbiter it will select data alternatively from channels 0 and 1 if both are active, which may prevent burst mode, and the WB Master does only 1 request per transfer cycle.     

There is also another option: The WISHBONE Public Domain Library for VHDL offers a DMA Module of its own.
It is intended only for educational and benchmarking purposes, and is thus very simple, but it can be adapted for more serious work.
It supports both single transfers and burst transfers.
However, it only does 32-bit transfers, thus it has the same issues as the DMA Module we have implemented, regarding the 128-bit bus.
Its initiation is vaguely described in the Library manual, and there are no description of any slave interface, which implies that adapting this module will require expantion with a slave interface, similar to the one implemented for this project, in order to function on SHMAC
\cite{WBLibrary}.