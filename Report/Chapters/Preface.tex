\chapter*{Abstract}

Recent years have seem the emergence of a new class of currencies, called
cryptocurrencies. These currencies use cryptography to provide security
and peer-to-peer networking to provide a decentralized system. Cryptography
is also an important part of the peer-to-peer network of the currencies,
in which a cryptographic hash with an arithmetic value beneath a certain
threshold is used to sign a valid block of transactions in the distributed
ledger. The process of finding such a hash is called ``mining''. The person
who successfully mines a block is rewarded according to the rules of the
particular cryptocoin being mined.

Bitcoin is the first and most popular of these currencies. It uses a
two-pass SHA-256 hash to sign its blocks. Huge amounts of computational
power is required to find a valid hash, and as such there has been a lot
of development in this area. Current-generation custom chips, ASICs, used
in bitcoin ``mining'' provide huge amounts of computing power, but are starting
to be limited by the available cooling and power resources.

To alleviate this problem, this thesis looks at the possibilities of using
heterogeneous computing to reduce power consumption and produce a more
power-efficient mining solution.

A SHA-256 accelerator and a DMA is developed and integrated into a tile for
the Single-ISA Heterogeneous MAny-core Computer, SHMAC, and a system with
multiple cores is tested in order to find out how performance scales and
what kind of performance and energy efficiency one can expect from such a system.

% Change below?

The results show that using an accelerator provides about 3,5 times better
performance than calculating the hashes in software. They also show that
using a DMA further increases performance. The performance does, however,
not reach the neccessary number of hashes per second to be fast enough to
be used for mining bitcoins. The testing also reveals some limitations of
SHMAC's memory system.

\chapter*{Sammendrag}

I løpet av de siste årene har det oppstått en ny klasse med valuater,
kalt ``cryptocurrencies'', eller ``digitale valuater'' på norsk. Disse
valuatene bruker kryptografi til å tilby sikkerhet og et peer-to-peer-nettverk
for å tilby et desentralisert system. Kryptografi er også en viktig del av
peer-to-peer-nettverket til valuatene, hvor en kryptografisk hash-funksjon
med en aritmetisk verdi under en viss grense blir brukt til å signere
gyldige blokker med transaksjoner i en distribuert liste som inneholder
alle transaksjonene som er blitt gjort i nettverket. Jobben med å finde
en slik hash kalles for ``mining''. Den som finner en gyldig hash for en
blokk blir tildelt en belønning, hvis størrelse avhenger av reglene for
det aktuelle nettverket.

Bitcoin er den første og mest populære digitale valuate. Den bruker en
dobbel SHA-256-hash for å signere blokkene sine. Store mengder med
datakraft er nødvendig for å finne en gyldig hash, og det har derfor
vært mye utvikling på dette feltet. Dagens generasjon av spesialiserte
databrikker, ASICs, brukt i bitcoin-mining tilbyr store mengder
datakraft, men begynner å bli begrenset av den tilgjengelige mengden
med kraft- og kjøleressurser.

For å bøte på dette problemet ser vi på mulighetene for å bruke heterogene
datamaskiner til å senke strømforbruket og skape en mer energieffektiv
mining-løsning.

En SHA-256 akselerator og en DMA blir utviklet og integrert i en tile
for SHMAC, the Single-ISA Heterogeneous MAny-core Computer, og et system
med flere kjerner blir testet for å finne ut hvordan ytelsen skalerer
og hvor stor ytelse og energieffektivitet man kan forvente fra et slikt system.

% Kanskje endre nedenfor?

Resultatene viser at å bruke en akselerator for hashing gir omtrent 3,5 ganger
bedre ytelse enn å gjøre den tilsvarende operasjonen i software. De viser òg
at å bruke en DMA ytterligere øker ytelsen. Ytelsen når imidlertid ikke den
nødvendige ytelsen for at systemet skal være kjappt nok til å kunne delta
i bitcoin-nettverket, og testingen avslører også enkelte begrensninger i
SHMACs minnesystem.

\chapter*{Acknowledgments}

% Can be changed :-)

We would like to thank Donn Morrison and Yaman Umurogly for their work as
supervisors on this project, taking time to assist us with all our problems.

In addition we would like to thank Asbjørn Djupdal for assistance in setting
up our Versatile Express box.

