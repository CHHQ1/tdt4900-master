\chapter*{Abstract}

Recent years have seem the emergence of a new class of currencies, called
cryptocurrencies. These currencies use cryptography to provide security
and peer-to-peer networking to provide a decentralized system. Bitcoin is
the most popular of these currencies. It uses a two-pass
SHA-256 hash at its core. Producing new bitcoins is done through a process
referred to as ``mining'', which involves a brute-force search for a hash with
a specific value. This process requires large amounts of computing power.

Current-generation hardware for bitcoin mining includes highly-optimized
ASIC chips which provide huge amounts of performance. However, designers of
such chips are having problems with delivering enough power and cooling
to the chips. To alleviate this problem, this thesis looks at the possibilities
of using heterogeneous computing to reduce power consumption and produce a more
energy-efficient mining solution.

A SHA-256 accelerator and a DMA is developed and integrated into a tile for
the Single-ISA Heterogeneous MAny-core Computer, SHMAC, and a system with
multiple cores is used to exploit the thread-level parallelism provided by
the platform. The system is tested using a benchmark to find out what performance
and energy efficiency can be expected when using the system for bitcoin mining.

The results show a maximum performance of 175,7~kH/s when running the benchmark
application on 14 cores using the SHA-256 accelerator and the DMA. The best
energy efficiency was obtained when running on 14 cores \emph{without} the DMA enabled,
at 163,2~kH/J. The results does not compare well to specialized FPGA-based
bitcoin miners, but demonstrates the SHMAC platform's large degree of thread-level parallelism
which can be better exploited in other applications.

\chapter*{Sammendrag}

De siste årene har en ny type valuta, kalt ``cryptocurrencies'' eller digitale valutaer. Disse valutaene
bruker kryptografi til å tilby sikkerhet og peer-to-peer-nettverk til å tilby et
desentralisert system. Bitcoin er den mest populære av dem. Bitcoin er bygd rundt
en dobbel SHA-256 hash. Å lage nye bitcoins blir gjort via en prosess som kalles
for ``mining'', og invovlerer et brute-force søk etter en hash med en bestemt verdi.
Dette krever store mengder med datakraft.

Den nåværende genersasjonen av hardware for bitcoin mining består av høyt optimaliserte
ASIC-brikker som gir store mengder ytelse. Likevel opplever designerne av slike brikker
problemer med kjøling og strømtilførsel. For å bøte på dette problemet ser vi på mulighetene
for å bruke heterogene arkitekturer for å redusere strømforbruk og lage en mer energieffektiv
løsning.

En SHA-256 akselerator og en DMA blir utviklet og integrert i en tile for SHMAC,
the Single-ISA Heterogeneous MAny-core Computer, og et system med flere kjerner
blir brukt for å utnytte trådnivåparallelliteten som platformen tilbyr. Systemet
blir testet med et benchmark-program for å finne ut hvilken ytelse og energieffektivitet
som kan forventes når man bruker systemet til bitcoin-mining.

Resultatene viser en maksimal ytelse på 175,5~kH/s når man kjører benchmark-programmet
på 14 kjerner med SHA-256 akseleratoren og DMAen. Den beste energieffektiviteten
ble observert når programmet ble kjørt på 14 kjerner med DMA-modulen avskrudd,
på 163,2~kH/J. Resultatene kommer ikke godt ut når man sammenlikner med spesialiserte
FPGA-baserte bitcoin-minere, men demonstrerer likevel hvordan SHMAC-platformens høye
grad av trådnivåparallelitet kan utnyttes av andre programmer.

\chapter*{Acknowledgments}

% Can be changed :-)

We would like to thank Donn Morrison and Yaman Umuroğlu for their work as
supervisors on this project, taking time to assist us with all our problems.

In addition we would like to thank Asbjørn Djupdal for assistance in setting
up our Versatile Express box.

